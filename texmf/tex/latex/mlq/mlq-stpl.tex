%%
%% This is file `mlq-stpl.tex',
%% generated with the docstrip utility.
%%
%% The original source files were:
%%
%% template.dtx  (with options: `mlq,small')
%% 
%% $Id: template.dtx,v 1.61 2003/09/16 00:30:21 uwe Exp $
%% ====================================================================
\documentclass[mlq,fleqn]{w-art}
\usepackage{times}
\usepackage{w-thm}
%% By default the equations are consecutively numbered. This may be changed by
%% the following command.
%% \numberwithin{equation}{section}
%%
%%
%% The usage of multiple languages is possible.
%% \usepackage{ngerman}% or
%% \usepackage[english,ngerman]{babel}
%% \usepackage[english,french]{babel}
\usepackage[]{graphicx}
\begin{document}
%%    The information for the title page will be placed between
%%    \begin{document} and \maketitle. The order of most entries
%%    is determined by the class file and can not be changed by
%%    rearranging them. The maketitle command follows after the
%%    abstract.
%%
%%    Most of the following commands will be completed by the publisher.
%%
%%    The copyrightyear is defined in the .clo file as the first argument
%%    of the copyrightinfo command. If the copyrightyear differs from that
%%    value it might be adjusted by the following definition:
%%
%% \renewcommand{\copyrightyear}{2003}% uncomment to change the copyrightyear.
%%
\DOIsuffix{malq.theDOIsuffix}
%%
%% issueinfo for header and copyright line
\Volume{49}
\Issue{49}
\Month{01}
\Year{2003}
%%
%%    First and last pagenumber of the article. If the option
%%    'autolastpage' is set (default) the second argument may be left empty.
\pagespan{3}{}
%%
%%    Dates will be filled in by the publisher. The 'reviseddate' and
%%    'dateposted' (Published online) entry may be left empty.
\Receiveddate{15 November 2003}
\Reviseddate{30 November 2003}
\Accepteddate{2 December 2003}
\Dateposted{3 December 2003}
%%
\keywords{List, of, comma, separated, keywords.}
\subjclass[msc2010]{04A25}

%% \pretitle{Editor's Choice}

%% We have a short and a long form for the title. The short form
%% (optional argument) goes into the running head.

\title[Short Title]{Long title}

%% Please do not enter footnotes or \inst{}-notes into the optional
%% argument of the author command. The optional argument will go into
%% the header.  If there is only one address the marker \inst{x} may be
%% omitted.

%% Information for the first author.
\author[Sh. First Author]{L. First Author\footnote{Corresponding
     author: e-mail: {\sf x.y@xxx.yyy.zz}, Phone: +00\,999\,999\,999,
     Fax: +00\,999\,999\,999}\inst{1}} \address[\inst{1}]{First address}
%%
%%    Information for the second author
\author[Sh. Second Author]{L. Second Author\footnote{Second author footnote.}\inst{1,2}}
\address[\inst{2}]{Second address}
%%
%%    Information for the third author
\author[Sh. Third Author]{L. Third Author\footnote{Third author footnote.}\inst{2}}
%%
%%    \dedicatory{This is a dedicatory.}
\begin{abstract}
\end{abstract}
%% maketitle must follow the abstract.
\maketitle                   % Produces the title.

%% If there is not enough space inside the running head
%% for all authors including the title you may provide
%% the leftmark in one of the following three forms:

%% \renewcommand{\leftmark}
%% {First Author: A Short Title}

%% \renewcommand{\leftmark}
%% {First Author and Second Author: A Short Title}

%% \renewcommand{\leftmark}
%% {First Author et al.: A Short Title}

%% \tableofcontents  % Produces the table of contents.
\section{First section}
\subsection{First subsection}




\begin{acknowledgement}
  An acknowledgement may be placed at the end of the article.
\end{acknowledgement}

The style of the following references should be used in all documents.

\begin{thebibliography}{10}
\bibitem{aa} xxx.
\end{thebibliography}

\end{document}               % End of document.

\endinput
%%
%% End of file `mlq-stpl.tex'.
